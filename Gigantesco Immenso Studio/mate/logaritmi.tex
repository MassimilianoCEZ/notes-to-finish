\documentclass{article}
\usepackage[utf8]{inputenc}
\usepackage{amsmath}
\usepackage{parskip}
\usepackage{dsfont}
\usepackage{fullpage}

\title{Math}

\author{Massimiliano Ferrulli}
\date{}

\begin{document}

\maketitle

\section*{Abstract}
Abstract

\pagebreak

\tableofcontents
\pagebreak

\section{Logaritmi}
\subsection{funzioni di tipo esponenziale e crescite di tipo esponenziale}
\begin{align*}  
    x(t)=&\,a \cdot b^\frac{t}{T}
    \\t =&\,periodo
    \\T =&\,tempo
    \\a =&\,accrescita
    \\a \in& \, \mathds{R}^+ \symbol{92}\,\{0\} 
    \\b \in& \, \mathds{R} \symbol{92}\,\{0\} 
\end{align*}

\begin{align*}
f(t+\Delta t)-f(t) 
\end{align*}
\begin{center}
Non è costante
\end{center}
\begin{align*}
 \frac{f(t+\Delta t)-f(t)}{f(t)} \cdot 100\%
\end{align*}
\begin{center}
È costante
\\ crescita relativa in un intervallo di tempo
\end{center}

\begin{align*}
    a^x 
\end{align*}    
\begin{center}   
è iniettiva allora se
\end{center}
\begin{align*}
         a^f(x) =& a^g(x)
        \\ f(x) =& g(x)
\end{align*}
\begin{center}
    la funzione inversa di una funzione iniettiva, è un'altra iniettiva
\end{center}
\subsubsection{funzione esponenziale}
\begin{align*}
    y=a^x
    \\y=x+1
\end{align*}
\begin{center}
    Queste due funzioni intersecano sempre in (0:1) e in un altro punto
    \\ C'è un valore di a che permette alle due funzioni di avere 
    \\ (0:1) come unico punto di intersezione, questo valore è e (numero di Eulero)
    \\ e le rende dangenti in (0;1), pendenza = 1
\end{center}
\begin{align*}
    e\cong 2,718281828459
\end{align*}
\begin{center}
    calcolo di e
    \\punto intersezione per ogni valore di a = (0;1)
    \\secondo punto di intersezione:
\end{center}
\begin{align*}
        \\ (\frac{1}{n};&y)
        \\ x + 1 = y \,|\, x = \frac{1}{n} \,|\, n \rightarrow \infty & \,|\, y = 1 \,|\, a_n^x = y  
        \\ a_n = (1 & + \frac{1}{n})^n
\end{align*}
\begin{center}
    limite per successione 
\end{center}
\begin{align*}
    e = \lim_{n \to \infty} (1 + \frac{1}{n})^n \,|\, e = 1 + \sum_{n = 1}^{\infty} 
      \frac{1}{k!} 
\end{align*}
\begin{center}
    e è irrazionale trascendente


\end{center}

\subsection{Logaritmi}










\end{document}