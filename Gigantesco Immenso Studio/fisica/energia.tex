\documentclass{article}
\usepackage[utf8]{inputenc}
\usepackage{amsmath}
\usepackage{parskip}
\usepackage{dsfont}
\usepackage{fullpage}

\title{Physics}
\author{Ferrulli Massimiliano}
\date{}

\begin{document}

\maketitle

\section*{Abstract}
Appunti realizzati da Ferrulli Massimiliano
\\
Every notes has been made by Ferrulli Massimiliano
\pagebreak

\tableofcontents
\pagebreak

\section{Energy}
First let's talk about Energy and what it is.
\\
Energy is something that can't be crated or destroyed in any way, the total amount is always the same.
However energy can be transformed from a type of energy to another one, but we'll talk later about types of energy.

\section{Kinetik Energy}
The Kinetik Energy \textit{\text{K}} is related to the state of motion of an object. the formula of the Kinetik Energy is:

\begin{align*}  
\textit{K} = \frac{1}{2} \, \textit{m} \, v^2
\end{align*}

\begin{center} 
The SI unit of \textit{K} is \textit{J} 
\end{center}
\section{Work}
Work is energy transferred from or to an object, Work is made by a force that acts on it.
when energy is transferred from an object to another system or object, the force does a negative work.
The work done from a point \textit{A} to a point \textit{B} is not affected by the path chosen.
\begin{align*}  
E_f - E_i = W
\end{align*}

Obviously if \(E_i>E_f\) we will have a negative work done by the force as part of the energy is 
transferred from a system to another one. 
\subsection{Advanced Part}
\begin{align*}
\textit{W} = \int_{x_i}^{x_f}  \,(x)\, dx    
\end{align*}
That's the formula of the \textit{W} done by any force.
Newtwon says that \(\textit{F} = \textit{m} \cdot \textit{a}\) and we can apply this to the formula written above.
\begin{align*}
\textit{W} = \int_{x_i}^{x_f}  \,F(x) \, dx &= \int_{x_i}^{x_f} \,\textit{m} \cdot \textit{a} \, dx
\\ 
\textit{a} &= \frac{dv}{dt}
\\
\textit{m} \cdot \textit{a} \, dx &= \textit{m} \cdot \frac{dv}{dt} \, dx    
\end{align*}
For the chain rule:
\begin{align*}
    \frac{dv}{dt} \, dx = \frac{dv}{dt} \cdot \frac{dx}{dx} = \frac{dv}{dx} \cdot \frac{dx}{dt} = \frac{dv}{dx} \cdot \textit{v}
\end{align*}
\begin{center}
By just changing what we've done before:
\end{center}
\begin{align*}
    \textit{m} \cdot \frac{dv}{dt} \, dx = \textit{m} \cdot \frac{dv}{dx} \, dx \cdot \textit{v} = \textit{m} \cdot \textit{v} \, dv
\end{align*}
if we put:
\begin{align*}
    \textit{m} \cdot \textit{a} \, dx = \textit{m} \cdot \textit{v} \, dv   
\end{align*}
inside
\begin{align*}
    \int_{x_i}^{x_f} \,\textit{m} \cdot \textit{a} \, dx   
\end{align*}
we have:
\begin{align*}
    \int_{v_i}^{v_f} \,\textit{m} \cdot \textit{v} \, dv = \frac{1}{2} \, \textit{m} \, v_f^2 - \frac{1}{2} \, \textit{m} \, v_i^2     
\end{align*}

\section{Potential Energy}
\begin{align*}
sniggers
\end{align*}

\end{document}
